\documentclass[english]{article}
\usepackage[a4paper,top=25mm,bottom=35mm,textwidth=160mm]{geometry}
\usepackage[T1]{fontenc}
\usepackage[dvips]{graphicx}
\usepackage{amssymb}
\usepackage{amsmath}
\usepackage[usenames]{color}
\usepackage[utf8]{inputenc}
%\usepackage{graphics}
\usepackage{babel}
\input{epsf}

\begin{document}
\newcommand{\mc}{\multicolumn}
\newcommand{\mr}{\multirow}
\newcommand{\cw}{\columnwidth}
\newcommand{\ig}[2]{\includegraphics[width=#1\cw]{#2}}

\title{Method and proof for Problem 133}
\author{I\~naki Silanes}
\maketitle

\section{Problem definition}

A number consisting entirely of ones is called a repunit. We shall define $R(k)$ to be a repunit of length $k$; for example, $R(6) = 111111$.\\

Let us consider repunits of the form $R(10^n)$.\\

Although $R(10)$, $R(100)$, or $R(1000)$ are not divisible by 17, $R(10000)$ is divisible by 17. Yet there is no value of $n$ for which $R(10^n)$ will divide by 19. In fact, it is remarkable that 11, 17, 41, and 73 are the only four primes below one-hundred that can be a factor of $R(10^n)$.\\

Find the sum of all the primes below one-hundred thousand that will never be a factor of $R(10^n)$.

\section{Method and proof}

Let's take prime $p$, and define $A(p) = k$ as the smallest $k$, such that $R(k) \propto p$, where $R(n)$ is the base-10 repunit of $n$ digits. Let's assume $R(q) \propto p$, where $q$ is the next smallest $n$ for which $R(n)$ is divisible by $p$. Then:

\begin{eqnarray}
      R(q) & = & R(k) + 10^k R(q-k) \propto p \\
R(q) \bmod p & = & ( R(k) \bmod p + (10^k R(q-k)) \bmod p ) \bmod p = 0 \\
R(q) \bmod p & = & (0 + (10^k R(q-k)) \bmod p ) \bmod p = 0 \\
R(q) \bmod p & = & (10^k R(q-k)) \bmod p = 0 \label{eq:qfirst}
\end{eqnarray}\\

From Eq.~\ref{eq:qfirst} we conclude that either $10^k \bmod p = 0$ or $R(q-k) \bmod p = 0$. We will discard the former, as follows:

\begin{eqnarray}
 R(k) = \frac{10^k - 1}{9} & \propto & p \\
10^k - 1 & = & n\cdot p \\
10^k & = & n\cdot p + 1 \\
10^k \bmod p & = & 1 \label{eq:tenkmodp}
\end{eqnarray}\\

So, from Eqs.~\ref{eq:qfirst} and \ref{eq:tenkmodp} we conclude that $R(q-k) \bmod p = 0$. Recall that $q$ is the second-smallest $n$ for $R(n) \bmod p = 0$, with $k$ being the smallest. Clearly $q-k$ is a valid $n$ for $R(n) \bmod = 0$, and is smaller than $q$. The only value smaller than $q$ with that property is $k$, so $q - k = k$, or $q = 2k$.\\

Repeating the same argument for the third and following smallest $n$ for $R(n) \bmod p = 0$, we conclude that if $R(q) \bmod p = 0$, then $q \propto k = A(p)$.\\

\begin{eqnarray}
R(q) \bmod p = 0 \implies q \bmod A(p) = 0 \label{eq:base}
\end{eqnarray}\\

Going back to the definition of the problem, and applying Eq.~\ref{eq:base}, if we assume that $R(10^n) \bmod p = 0$ for some $n$, then it follows that $10^n \bmod A(p) = 0$. In other words, $A(p)$ must be of the form $2^i\cdot 5^j$. Any $p$ for which $A(p)$ is not of that form will never be a divisor of $R(10^n)$, for {\bf any} $n$.\\

\subsection{f0}

The method for solving the problem, then, will consist on looping over all primes $p$ below $10^5$, finding $A(p)$, and finding out whether $A(p) = 2^i\cdot 5^j$ for integer $i$ and $j$.

\subsection{f1}

Same as {\bf f0}, but calculate $A(p)$ using $10^k \bmod p = 1$, instead of $R(k) \bmod p = 0$, which is slightly faster.

\subsection{f2}

Instead of finding $A(p)$ for each $p$, and {\it then} checking whether $A(p) \bmod 2^i\cdot 5^j = 0$, only check all $k = 2^i\cdot 5^j < p$, for $R(k) \bmod p = 0$ (that is $10^k \bmod p = 1$). If no such $k$ exists, then we don't know what $A(p)$ is for $p$, but we do know that its prime factors won't be just 2 and 5.

\end{document}
