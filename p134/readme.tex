\documentclass[english]{article}
\usepackage[a4paper,top=25mm,bottom=35mm,textwidth=160mm]{geometry}
\usepackage[T1]{fontenc}
\usepackage[dvips]{graphicx}
\usepackage{amssymb}
\usepackage[usenames]{color}
\usepackage[utf8]{inputenc}
%\usepackage{graphics}
\usepackage{babel}
\input{epsf}

\begin{document}
\newcommand{\mc}{\multicolumn}
\newcommand{\mr}{\multirow}
\newcommand{\cw}{\columnwidth}
\newcommand{\ig}[2]{\includegraphics[width=#1\cw]{#2}}

\title{Problem 134}
\author{I\~naki Silanes}
\maketitle

\section{Definition}

Consider the consecutive primes $p_1 = 19$ and $p_2 = 23$. It can be verified that 1219 is the smallest number such that the last digits are formed by $p_1$ whilst also being divisible by $p_2$.\\

In fact, with the exception of $p_1 = 3$ and $p_2 = 5$, for every pair of consecutive primes, $p_2 > p_1$, there exist values of $n$ for which the last digits are formed by $p_1$ and $n$ is divisible by $p_2$. Let $S$ be the smallest of these values of $n$.\\

Find $\sum S$ for every pair of consecutive primes with $5 \leq p_1 \leq 1000000$.

\section{Solution}

The requested $n$ will always be of the form $n = m 10^d + p_1$, where $d$ is the amount of digits in $p_1$. The requested property for $n$ is

\begin{eqnarray}
n \bmod p_2 & = & 0\\
(m 10^d + p_1) \bmod p_2 & = & 0\\
(m 10^d \bmod p_2 + p_1 \bmod p_2) \bmod p_2 & = & 0\\
(m 10^d \bmod p_2 + p_1) \bmod p_2 & = & 0\\
m 10^d \bmod p_2 & = & p_2 - p_1 \label{eq:mod}
\end{eqnarray}

Following this reasoning, we proceed to check different $m$ values for each $p_1, p_2$ pair, until Eq.~\ref{eq:mod} holds. Then we calculate $n = m 10^d + p_1 = S$ and add it to the total.

\end{document}
