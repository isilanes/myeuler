\documentclass[english]{article}
\usepackage[a4paper,top=25mm,bottom=35mm,textwidth=160mm]{geometry}
\usepackage[T1]{fontenc}
\usepackage[dvips]{graphicx}
\usepackage{amssymb}
\usepackage[usenames]{color}
\usepackage[utf8]{inputenc}
%\usepackage{graphics}
\usepackage{babel}
\input{epsf}

\begin{document}
\newcommand{\mc}{\multicolumn}
\newcommand{\mr}{\multirow}
\newcommand{\cw}{\columnwidth}
\newcommand{\ig}[2]{\includegraphics[width=#1\cw]{#2}}

\title{Problem 134}
\author{I\~naki Silanes}
\maketitle

\section{Definition}

Given the positive integers, $x$, $y$, and $z$, are consecutive terms of an arithmetic progression, the least value of the positive integer, $n$, for which the equation, $x^2 - y^2 - z^2 = n$, has exactly two solutions is $n = 27$:

\begin{equation}
34^2 - 27^2 - 20^2 = 12^2 - 9^2 - 6^2 = 27
\end{equation}

It turns out that $n = 1155$ is the least value which has exactly ten solutions.\\

How many values of $n$ less than one million have exactly ten distinct solutions?

\section{Solution(s) and proof}

aaa

\end{document}
