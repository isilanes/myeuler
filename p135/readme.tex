\documentclass[english]{article}
\usepackage[a4paper,top=25mm,bottom=35mm,textwidth=160mm]{geometry}
\usepackage[T1]{fontenc}
\usepackage[dvips]{graphicx}
\usepackage{amssymb}
\usepackage[usenames]{color}
\usepackage[utf8]{inputenc}
%\usepackage{graphics}
\usepackage{babel}
\input{epsf}

\begin{document}
\newcommand{\mc}{\multicolumn}
\newcommand{\mr}{\multirow}
\newcommand{\cw}{\columnwidth}
\newcommand{\ig}[2]{\includegraphics[width=#1\cw]{#2}}

\title{Problem 135}
\author{I\~naki Silanes}
\maketitle

\section{Definition}

Given the positive integers, $x$, $y$, and $z$, are consecutive terms of an arithmetic progression, the least value of the positive integer, $n$, for which the equation, $x^2 - y^2 - z^2 = n$, has exactly two solutions is $n = 27$:

\begin{equation}
34^2 - 27^2 - 20^2 = 12^2 - 9^2 - 6^2 = 27
\end{equation}

It turns out that $n = 1155$ is the least value which has exactly ten solutions.\\

How many values of $n$ less than one million have exactly ten distinct solutions?

\section{Solution(s) and proof}

If ${x,y,z}$ are members of an arithmetic progression, then we can define, without losing generality, $x = y + d$ and $z = y - d$, for some integer $d < y$ (lest $z \leq 0$). So:

\begin{eqnarray}
x^2 - y^2 - z^2 & = & n \\
(y+d)^2 - y^2 - (y-d)^2 & = & n \\
4 d y - y^2 & = & n \label{eq:n}
\end{eqnarray}

Solving for $y$, we get:

\begin{equation}
y = 2d \pm \sqrt{4d^2-n} \label{eq:y}
\end{equation}

From Eq.~\ref{eq:y} we can obtain the maximum $d$ possible. We know that the content of the square root must be a perfect square, so, defining $D = 2d$:

\begin{eqnarray}
4d^2 - n & = & k^2 \\
D^2 - n & = & k^2 \label{eq:D2}
\end{eqnarray}

For Eq.~\ref{eq:D2} to hold, $D^2 - n$ must be equal to or less than $(D-1)^2$, since $D-1$ is the largest value $k$ could take. Developing further:

\begin{eqnarray}
D^2 - n & \leq & (D-1)^2 \\
D^2 - n & \leq & D^2 + 1 -2D \\
n & \geq & 2D-1 = 4d-1 \\
d & \leq & \frac{n+1}{4}
\end{eqnarray}

Now, for a given $d$, what would the limits for $y$ be? Since $z = y-d > 0$, then $y \geq d + 1$. Also, from Eq.~\ref{eq:n}:

\begin{eqnarray}
4 d y - y^2 & = & n > 0 \\
y(4d-y) & > & 0 \\
4d - y & > & 0 \\
y & < & 4d
\end{eqnarray}

\subsection{Solution f3}

The procedure to solve this problem would then be the following:

\begin{enumerate}
  \item Take all $d$ from 1 to $(n_{max}+1)/4$
  \item Take all $y$ from $d+1$ to $4d$ (see above)
  \item Calculate $n$ from Eq.~\ref{eq:n}
  \item If $n$ is within 0 and $n_{max}$, add 1 to the amount of combinations that yield $n$
  \item Once all $(d, y)$ taken, check all $n$ to see if its amount of combinations is 10, and print how many of them are
\end{enumerate}

\subsection{Solution f5}

We must take into account that in the $y = d+1$ to $y = 4d$ region there can be a sizeable region where $n > n_{max}$ for sure. See that Eq.~\ref{eq:n} is a parabola, when plotting $n$ vs. $y$ for a given $d$. Its maximum will be at $y = 2d$, with a value of $n = 4d^2$. If this value is less than $n_{max}$ all $y$ values in interval will yield a valid $n$. However, if $n = 4d > n_{max}$, there will be a $y = 2d \pm \delta y$ region around the maximum where we do not need to check $y$, because we know it will yield too large an $n$.\\

The width of that region, $\delta y$ can be obtained from Eq.~\ref{eq:y}, substituting $n$ with $n_{max}$, and turns out to be $\delta y = \sqrt{4d^2-n_{max}}$.\\

Taking advantage of this fact, we can loop over $y$ only in the $(d+1, 2d - \delta y)$ and $(2d + \delta y, 4d)$ regions. For larger $d$ values this saves quite a bit of time.

\end{document}
