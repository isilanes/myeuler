\documentclass[english]{article}
\usepackage[a4paper,top=25mm,bottom=35mm,textwidth=160mm]{geometry}
\usepackage[T1]{fontenc}
\usepackage[dvips]{graphicx}
\usepackage{amssymb}
\usepackage[usenames]{color}
\usepackage[utf8]{inputenc}
%\usepackage{graphics}
\usepackage{babel}
\usepackage{charter}
\input{epsf}

\begin{document}
\newcommand{\mc}{\multicolumn}
\newcommand{\mr}{\multirow}
\newcommand{\cw}{\columnwidth}
\newcommand{\ig}[2]{\includegraphics[width=#1\cw]{#2}}

\title{Problem 138\\Special isosceles triangles}
\author{I\~naki Silanes}
\maketitle

\section{Definition}

Consider the isosceles triangle with base length, $b = 16$, and legs, $L = 17$.\\

By using the Pythagorean theorem it can be seen that the height of the triangle, $h = \sqrt{17^2 - 8^2} = 15$, which is one less than the base length.\\

With $b = 272$ and $L = 305$, we get $h = 273$, which is one more than the base length, and this is the second smallest isosceles triangle with the property that $h = b \pm 1$.\\

Find $\sum L$ for the twelve smallest isosceles triangles for which $h = b \pm 1$ and $b$, $L$ are positive integers.


\section{Solution(s) and proof}

Defining $B = b/2$ without losing generality, since $b$ must be even, we have:

\begin{eqnarray}
L^2 & = & B^2 + h^2 \\
L^2 & = & B^2 + (B \pm 1)^2 \\
5B^2 \pm 4B + 1 - L^2 & = & 0\label{eq:L2}
\end{eqnarray}

Solving for $B$, we get:

\begin{eqnarray}
B = \frac{\pm 2 \pm \sqrt{5 L^2 - 1}}{5} \label{eq:B}
\end{eqnarray}

We can safely remove the negative square root solution from Eq.~\ref{eq:B}, since $B > 0$.\\

\subsection{f0}

Eq.~\ref{eq:L2} already gives as a method for solving p138. We can try succesive integer values of $B$, and check whether the result of $5B^2\pm 4B+1$ is a perfect square. If it is, calculate $L$ and add up. This method is too slow to go beyond the 7th triangle.

\subsection{f1}

According to Eq.~\ref{eq:B}, we can take succesive $L$ values, square them, then check whether the equation returns an integer value for $B$. This method turns out to be a bit slower than {\bf f0}.

\subsection{f2}

I have realized that the values of $L$ in Eq.~\ref{eq:B} that return an integer $B$ are exactly half the value of members of the Fibonacci sequence, more precisely of the form $F_{9+6m}/2$ for $m = 0, 1, 2, 3...$ It is then trivial to iterate over every 6th Fibonacci number from the 9th on, adding their halves up as needed.

\end{document}
