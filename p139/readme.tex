\documentclass[english]{article}
\usepackage[a4paper,top=25mm,bottom=35mm,textwidth=160mm]{geometry}
\usepackage[T1]{fontenc}
\usepackage[dvips]{graphicx}
\usepackage{amssymb}
\usepackage[usenames]{color}
\usepackage[utf8]{inputenc}
%\usepackage{graphics}
\usepackage{babel}
\usepackage{charter}
\input{epsf}

\begin{document}
\newcommand{\mc}{\multicolumn}
\newcommand{\mr}{\multirow}
\newcommand{\cw}{\columnwidth}
\newcommand{\ig}[2]{\includegraphics[width=#1\cw]{#2}}

\title{Problem 139\\Pythagorean tiles}
\author{I\~naki Silanes}
\maketitle

\section{Definition}

Let $(a, b, c)$ represent the three sides of a right angle triangle with integral length sides. It is possible to place four such triangles together to form a square with length $c$.\\

For example, $(3, 4, 5)$ triangles can be placed together to form a 5 by 5 square with a 1 by 1 hole in the middle and it can be seen that the 5 by 5 square can be tiled with twenty-five 1 by 1 squares.\\

However, if (5, 12, 13) triangles were used then the hole would measure 7 by 7 and these could not be used to tile the 13 by 13 square.\\

Given that the perimeter of the right triangle is less than one-hundred million, how many Pythagorean triangles would allow such a tiling to take place?

\section{Solution(s) and proof}

We will call $d$ the value of the side of the hole in the middle, and $(i, j, k)$ the values of the sides of the triangle, in increasing order. From that is evident that $j - i = d$, and $i^2 + j^2 = k^2$. Also, as $k$ must be divisible by $d$, we can define $k = d k_0$. From that:\\

\begin{eqnarray}
i^2 + j^2 & = & k^2 \\
i^2 + (j+d)^2 & = & d^2 k_0^2 \\
2 i^2 + 2id + d^2(1-k_0^2) & = & 0 \\
\end{eqnarray}

Solving for $i$, we get:

\begin{eqnarray}
i & = & \frac{d}{2}(-1 + \sqrt{2k_0^2 - 1}) = i_0 d \\
i_0 & = & \frac{-1 + \sqrt{2k_0^2 - 1}}{2}
\end{eqnarray}

Since $j = i + d$, we can put it as a function of $i_0$ and $d$: $j = (i_0 + 1) d$. From that, we can solve for the perimeter:

\begin{eqnarray}
P & = & i + j + k \\
P & = & i_0 d + (i_0+1)d + k_0 d \\
P & = & d (2 i_0 + k_0 +1) \\
P & = & P_0 d
\end{eqnarray}

As we know that $P < P_{max}$ must hold, we get:

\begin{eqnarray}
\end{eqnarray}

\subsection{f0}

Eq.~\ref{eq:L2} already gives as a method for solving p138. We can try succesive integer values of $B$, and check whether the result of $5B^2\pm 4B+1$ is a perfect square. If it is, calculate $L$ and add up. This method is too slow to go beyond the 7th triangle.

\subsection{f1}

According to Eq.~\ref{eq:B}, we can take succesive $L$ values, square them, then check whether the equation returns an integer value for $B$. This method turns out to be a bit slower than {\bf f0}.

\subsection{f2}

I have realized that the values of $L$ in Eq.~\ref{eq:B} that return an integer $B$ are exactly half the value of members of the Fibonacci sequence, more precisely of the form $F_{9+6m}/2$ for $m = 0, 1, 2, 3...$ It is then trivial to iterate over every 6th Fibonacci number from the 9th on, adding their halves up as needed.

\end{document}
