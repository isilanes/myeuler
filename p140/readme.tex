\documentclass[english]{article}
\usepackage[a4paper,top=15mm,bottom=25mm,textwidth=160mm]{geometry}
\usepackage[T1]{fontenc}
\usepackage[dvips]{graphicx}
\usepackage{amssymb}
\usepackage[usenames]{color}
\usepackage[utf8]{inputenc}
%\usepackage{graphics}
\usepackage{babel}
\usepackage{charter}
\input{epsf}

\begin{document}
\newcommand{\mc}{\multicolumn}
\newcommand{\mr}{\multirow}
\newcommand{\cw}{\columnwidth}
\newcommand{\ig}[2]{\includegraphics[width=#1\cw]{#2}}

\title{Problem 140\\Modified Fibonacci golden nuggets}
\author{I\~naki Silanes}
\maketitle

\section{Definition}

Consider the infinite polynomial series $A_{G}(x) = x G_1 + x^2 G_2 + x^3 G_3 + ...$, where $G_k$ is the $k$th term of the second order recurrence relation $G_k = G_{k-1}+ G_{k-2}, G_1 = 1$ and $G_2 = 4$.\\

For this problem we shall be concerned with values of $x$ for which $A_G(x)$ is a positive integer.\\

The corresponding values of x for the first five natural numbers are shown below.\\

\begin{center}
  \begin{tabular}{cc} \hline
	$x$                 & $A_G(x)$ \\ \hline
	$(\sqrt{5}-1)/4$    & 1 \\
	$2/5$               & 2 \\
	$(\sqrt{22}-2)/6$   & 3 \\
	$(\sqrt{137}-5)/14$ & 4 \\
	$1/2$               & 5 \\ \hline
  \end{tabular}
\end{center}
\ \\

We shall call $A_G(x)$ a golden nugget if $x$ is rational, because they become increasingly rarer; for example, the 20th golden nugget is 211345365.\\

Find the sum of the first thirty golden nuggets.

\section{Solution(s) and proof}

It is evident that this is a modified version of problem 137. We will first derive a closed formula for $A_G(x)$, as the one we got from Wikipedia por $A_F(x)$.

\begin{eqnarray}
s(x) & = & \sum_{k=1}^{\infty} x^k G_k \\
s(x) & = & x G_1 + x^2 G_2 + \sum_{k=3}^{\infty} x^k G_k \\
s(x) & = & x + 4x^2 + \sum_{k=3}^{\infty} x^k (G_{k-1} + G_{k-2}) \\
s(x) & = & x + 4x^2 + x \sum_{k=1}^{\infty} x^k G_k - x^2 G_1 + x^2 \sum_{k=1}^{\infty}) x^k G_k \\
s(x) & = & x + 4x^2 + x s(x) - x^2 + x^2 s(x) \\
s(x) & = & \frac{x+3x^2}{1-x-x^2} \label{eq:sx}
\end{eqnarray}

Our only task is to find values of $x$ such that $s(x)$ is integer. If we make $s(x) = n$ in Eq.~\ref{eq:sx}, and solve for $x$:

\begin{eqnarray}
n & = & \frac{x + 3x^2}{1-x-x^2} \\
n (1-x-x^2) & = & x + 3x^2\\
(3+n)x^2 + (n+1) x - n & = & 0 \\
x & = & \frac{-(n+1)+\sqrt{5n^2+14n+1}}{2(3+n)} \label{eq:x}
\end{eqnarray}

In Eq.~\ref{eq:x} se remove the negative solution, as $x > 0$.

\subsection{f0}

Eq.~\ref{eq:x} already gives as a method for solving p140. We can try succesive integer values of $n$, and check whether the result of $5n^2+14n+1$ is a perfect square. If (and only if) it is, $x$ will be rational. This method is too slow to go beyond the 21st golden nugget.

\subsection{f1}

We can take Eq.~\ref{eq:x} and equate $5n^2+14n+1$ to some squared integer $k^2$, then solve for $n$:

\begin{eqnarray}
5n^2+14n+1 & = & k^2 \\
n & = & \frac{-7+\sqrt{5k^2+44}}{5} \label{eq:n}
\end{eqnarray}

According to Eq.~\ref{eq:n}, we can take succesive $k$ values, substitute them, then check whether the equation returns an integer value for $n$. This method turns out to be slower than {\bf f0}.

\subsection{f2}

I wondered whether the valid $k$ values for an integer $n$ in Eq.~\ref{eq:n} belong to any series, as in Problem 137 they belonged to the Fibonacci series.\\

The first 10 values of $k$ are: 7, 14, 50, 97, 343, 665, 2351, 4558, 16114, 31241. We can readily see that the even terms $(14, 97, 665, 4558, 31241)$, are members of the $G_i$, series, namely $G_i$ for $i~=~5, 9, 13, 17, 21$. In other words:\\

\begin{eqnarray}
k_i & = & G_{2i+1}
\end{eqnarray}

for even $i$.\\

However, the odd values are more puzzling: $(7,50,343,2351,16114, ...)$. Playing around, and searching a bit in internet, I found out they are members of another modified Fibonacci series, namely $H_i = H_{i-1} + H_{i-2}$, with $H_1 = 7$ and $H_2 = 12$. More precisely, they'd be elements $H_i$ for $i = 1, 5, 9, 13, 17, ...$. In other words:\\

\begin{eqnarray}
k_i & = & H_{2i-1}
\end{eqnarray}

for odd $i$.\\

We are requested to find the value of the first 30 golden nuggets, which amounts to finding $G_{2i+1}$ for $i = 2, 4, ..., 30$, then $H_{2i-1}$ for $i = 1, 3, ... 29$. These 30 values of $k$ will produce 30 values of $n$, via Eq.~\ref{eq:n}. Add up those 30 values of $n$, and be done with it.\\

No need to say this method is amazingly fast.

\end{document}
