\documentclass[english]{article}
\usepackage[a4paper,top=15mm,bottom=25mm,textwidth=160mm]{geometry}
\usepackage[T1]{fontenc}
\usepackage[dvips]{graphicx}
\usepackage{amssymb}
\usepackage[usenames]{color}
\usepackage[utf8]{inputenc}
%\usepackage{graphics}
\usepackage{babel}
\usepackage{charter}
\input{epsf}

\begin{document}
\newcommand{\mc}{\multicolumn}
\newcommand{\mr}{\multirow}
\newcommand{\cw}{\columnwidth}
\newcommand{\ig}[2]{\includegraphics[width=#1\cw]{#2}}

\title{Problem 141\\Investigating progressive numbers, n, which are also square}
\author{I\~naki Silanes}
\maketitle

\section{Definition}

A positive integer, $n$, is divided by $d$ and the quotient and remainder are $q$ and $r$ respectively. In addition $d$, $q$, and $r$ are consecutive positive integer terms in a geometric sequence, but not necessarily in that order.\\

For example, 58 divided by 6 has quotient 9 and remainder 4. It can also be seen that 4, 6, 9 are consecutive terms in a geometric sequence (common ratio 3/2).\\

We will call such numbers, $n$, progressive.\\

Some progressive numbers, such as 9 and $10404 = 102^2$, happen to also be perfect squares. The sum of all progressive perfect squares below one hundred thousand is 124657.\\

Find the sum of all progressive perfect squares below one trillion ($10^12$).

\section{Solution(s) and proof}

\subsection{f0}

We can take the most obvious route, and check all perfect squares below $10^{12}$, to see whether they are progressive or not. For each of those $10^6$ numbers $m$, we will check all potential divisors $d$, from $d=2$ to $d=\sqrt{m}$, obtaining $q$ and $r$ for each $d$. Then we will check whether they are members of a geometrical progression.\\

The progression check is easy, since by construction our values will be in $q > d > r$ order. This means we will have to check $q/d = d/r$, or reordering for better computation suitability: $q r = d^2$.\\

This method is usable, but too slow to be acceptable (around 5 hours on my PC).

\subsection{f1}

We can define $R$ as the coefficient of the geometric progression (not necessarily integer). By definition we have $r < d$, obviously, but theoretically $q$ could be either larger than $d$, between $d$ and $r$, or smaller than $r$. If the smallest of the three is $r$, then either $d = R r$ and $q = R d = R^2 r$ or $q = R r$ and $d = R q = R^2 r$. In both cases:

\begin{eqnarray}
n & = & d q + r\\
n & = & R^3 r^2 + r \label{eq:n1}
\end{eqnarray}

However, if the smallest one is $q$, then $r = Rq$ ($r$ must be the second largest, as $r < d$ by definition), and $d = R^2$, so:

\begin{eqnarray}
n & = & d q + r\\
n & = & R^2 q^2 + Rq \label{eq:n2}
\end{eqnarray}

We will argue that all the numbers we seek are of the form in Eq.~\ref{eq:n1}, and not Eq.~\ref{eq:n2}. Indeed, in the latter $Rq$ must be integer for $n$ to be integer. This means $n = c^2 + c$, with $n$ and $c$ integer. If $n$ is a square number, then $n = k^2$ for some integer $k$. So:

\begin{eqnarray}
k^2 & =  & c^2 + c \\
k & >  & c \\
k & \geq & c + 1 \\
k^2 & \geq & c^2 + 2c + 1 > c^2 + c = k^2 \label{eq:k2}
\end{eqnarray}

From Eq.~\ref{eq:k2}, it is clearly impossible that $n = c^2 + c$ be square.\\

So, without loss of generality, we can define $R = \alpha/\beta$, and use Eq.~\ref{eq:n1} to build up a method: 

\begin{enumerate}
  \item check all integer $r$ from 1 upwards
  \item From $r$, obtain all possible $d$ and $q$ pairs, from $\alpha r/\beta$ and $\alpha^2 r/\beta^2$
  \item If $d$ and $q$ are integer, use Eq.\ref{eq:n1} to get $n$
  \item If $n$ is integer (and smaller than limit), add it up
  \item If $n$ is larger than limit, skip to next $r$
  \item If $r^2$ is larger than limit, break from main loop
\end{enumerate}

This method is a bit slow, but acceptable ($\sim 5$ minutes).

\end{document}
