\documentclass[english]{article}
\usepackage[a4paper,top=15mm,bottom=25mm,textwidth=160mm]{geometry}
\usepackage[T1]{fontenc}
\usepackage[dvips]{graphicx}
\usepackage{amssymb}
\usepackage[usenames]{color}
\usepackage[utf8]{inputenc}
%\usepackage{graphics}
\usepackage{babel}
\usepackage{charter}
\input{epsf}

\begin{document}
\newcommand{\mc}{\multicolumn}
\newcommand{\mr}{\multirow}
\newcommand{\cw}{\columnwidth}
\newcommand{\ig}[2]{\includegraphics[width=#1\cw]{#2}}

\title{Problem 143\\Investigating the Torricelli point of a triangle}
\author{I\~naki Silanes}
\maketitle

\section{Definition}

Let ABC be a triangle with all interior angles being less than 120 degrees. Let X be any point inside the triangle and let $XA = p$, $XC = q$, and $XB = r$.\\

Fermat challenged Torricelli to find the position of X such that p + q + r was minimised.\\

Torricelli was able to prove that if equilateral triangles AOB, BNC and AMC are constructed on each side of triangle ABC, the circumscribed circles of AOB, BNC, and AMC will intersect at a single point, T, inside the triangle. Moreover he proved that T, called the Torricelli/Fermat point, minimises p + q + r. Even more remarkable, it can be shown that when the sum is minimised, $AN = BM = CO = p + q + r$ and that $AN$, $BM$ and $CO$ also intersect at $T$.\\

If the sum is minimised and a, b, c, p, q and r are all positive integers we shall call triangle ABC a Torricelli triangle. For example, $a = 399$, $b = 455$, $c = 511$ is an example of a Torricelli triangle, with $p + q + r = 784$.\\

Find the sum of all distinct values of $p + q + r \leq 120000$ for Torricelli triangles.

\section{Solution(s) and proof}

\subsection{f0}

Wrong

\subsection{f1}

We know that the 3 angles around $T$ (ATB, BTC, CTA) must be 120 degree angles. Check mathisfun.com\footnote{http://www.mathsisfun.com/geometry/circle-theorems.html}: either from the first inscribed angle theorem (inscribed angle is half of the central angle, 60 is half of 120), or from the first cyclic quadrilateral theorem (opposite angles add up to 180 degrees, $60 + 120 = 180$).\\

Applying the law of cosines to $r$, $q$ and $a$:\\

\begin{eqnarray}
a^2 &=& r^2 + q^2 - 2 r q \cos{120} = r^2 + q^2 + r q \label{eq:a2}
\end{eqnarray}

And, of course, the same holds for $(q,p,b)$ and $(p,r,c)$.\\

We can then solve the problem by taking integer $r$ values from 1 up and integer $q$ values from $r+1$ up. If they produce an integer $a$ value, according to Eq.~\ref{eq:a2}, proceed to check $p$ values from $q+1$ up. For each, check that $(q,p)$ produce an integer $b$, and $(p,r)$ an integer $c$. If so, add $p+q+r$ to the set of solutions.

\end{document}
