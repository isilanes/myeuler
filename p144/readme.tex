\documentclass[english]{article}
\usepackage[a4paper,top=15mm,bottom=25mm,textwidth=160mm]{geometry}
\usepackage[T1]{fontenc}
\usepackage[dvips]{graphicx}
\usepackage{amssymb}
\usepackage[usenames]{color}
\usepackage[utf8]{inputenc}
%\usepackage{graphics}
\usepackage{babel}
\usepackage{charter}
\input{epsf}

\begin{document}
\newcommand{\mc}{\multicolumn}
\newcommand{\mr}{\multirow}
\newcommand{\cw}{\columnwidth}
\newcommand{\ig}[2]{\includegraphics[width=#1\cw]{#2}}

\title{Problem 144\\Investigating multiple reflections of a laser beam}
\author{I\~naki Silanes}
\maketitle

\section{Definition}

In laser physics, a ``white cell'' is a mirror system that acts as a delay line for the laser beam. The beam enters the cell, bounces around on the mirrors, and eventually works its way back out.\\

The specific white cell we will be considering is an ellipse with the equation $4x^2 + y^2 = 100$\\

The section corresponding to $-0.01 \leq x \leq +0.01$ at the top is missing, allowing the light to enter and exit through the hole.\\

The light beam in this problem starts at the point $(0.0,10.1)$ just outside the white cell, and the beam first impacts the mirror at $(1.4,-9.6)$.\\

Each time the laser beam hits the surface of the ellipse, it follows the usual law of reflection ``angle of incidence equals angle of reflection''. That is, both the incident and reflected beams make the same angle with the normal line at the point of incidence.\\

In the figure on the left, the red line shows the first two points of contact between the laser beam and the wall of the white cell; the blue line shows the line tangent to the ellipse at the point of incidence of the first bounce.\\

The slope $m$ of the tangent line at any point $(x,y)$ of the given ellipse is: $m = -4x/y$\\

The normal line is perpendicular to this tangent line at the point of incidence.\\

The animation on the right shows the first 10 reflections of the beam.\\

How many times does the beam hit the internal surface of the white cell before exiting?\\

\section{Solution(s) and proof}

\subsection{f0}

We must first find the slope of the reflection of a beam with slope $m$ bouncing on the wall at a point $(x,y)$. Say the incoming beam has an angle $\alpha$ (all angles defined with respect to the X axis). So $\tan\alpha = m$. Also, the tangent to the wall at $(x,y)$ will have an angle of $\varphi$. As the problem definition gives us, the slope will be $\tan\varphi = -4x/y$. If we call $\gamma$ the angle between the tangent and the reflected beam, then the angle of the outgoing beam with respect to X will be $\psi = \varphi + \gamma$. With trigonometry we can solve that $\gamma = \varphi - \alpha$, so $\psi = 2\varphi - \alpha$. We know the values of the tangents of $\varphi$ and $\alpha$, and we seek the slope (tangent) of $\psi$. Using trigonometric relationships, and susbtituting the values for the tangents of $\varphi$ and $\alpha$, we obtain:

\begin{eqnarray}
\tan \psi &=& \frac{8xy - m(16x^2-y^2)}{16x^2-y^2+8xym} \label{eq:tany}
\end{eqnarray}

With that we can calculate the slope of the outgoing beam, with the slope of the incoming one, and the point of impact.\\

Then, we need to be able to find out the next reflection point of the outgoing beam. This beam will hit the wall where its path crosses the ellipse. The beam is a straight line, with the formula $y = A +mx$, where $A = y_0 - m x_0$, where $(x_0, y_0)$ is the last point the beam hit the wall. This line will cross the ellipse at:\\

\begin{eqnarray}
4x^2 + y^2 &=& 100 \\
4x^2 + (A+mx)^2 - 100 &=& 0\\
x &=& \frac{-Am \pm \sqrt{100m^2 + 400 - 4A^2}}{4+m^2} \label{eq:x}
\end{eqnarray}

So, given a beam with a slope $m$ and an impact point $(x,y)$, we can find the slope of the outgoing beam with Eq.~\ref{eq:tany}. Then, with the new slope $m = \tan\varphi$ and the previous impact point $(x_0,y_0)$ we can get the new impact point from Eq.~\ref{eq:x}. This equation gives to $x$ values (with the plus-minus). One of them will be $x_0$, so the other one will be the new $x$. From the new $x$ we get the new $y$, from $y = A + mx$, for example.\\

Repeating the procedure in the above paragraph we can find the successive impact points of the beam. Any time the impact point has $-0.01 \leq x \leq 0.01$ and $y$ is positive, the beam will escape the white cell and the computation will be over.

\end{document}
